\documentclass[handout,notheorems]{beamer}

\usepackage[T2A,T1]{fontenc}
\usepackage[utf8]{inputenc}
\usepackage[english]{babel}

\usepackage{tikz}
\usepackage{amsthm}
\newtheorem*{lemma}{Lemma}
\theoremstyle{definition}
\newtheorem*{definition}{Definition}

\hypersetup{unicode=true}

\usetheme{CambridgeUS}

\setbeamercolor{frametitle}{parent=palette primary}

\setbeamercolor{block title}{fg=black,bg=white!60!gray}
\setbeamercolor{block body}{fg=black,bg=white!90!gray}

\setbeamercolor*{author in head/foot}{parent=palette tertiary}
\setbeamercolor*{title in head/foot}{parent=palette secondary}
\setbeamercolor*{date in head/foot}{parent=palette primary}

\setbeamercolor*{section in head/foot}{parent=palette tertiary}
\setbeamercolor*{subsection in head/foot}{parent=palette primary}

\setbeamercolor{title}{bg=red!65!black,fg=white}
\setbeamersize{text margin left=1em,text margin right=1em}

% \AtBeginSection[]
% {
%     \begin{frame}{Table of Contents}
%         \tableofcontents[currentsection,hideothersubsections]
%     \end{frame}
% }

\institute[KNU]{Taras Shevchenko National University of Kyiv}
\author[Skybytskyi N.]{Nikita M. Skybytskyi, n.skybytskyi@knu.ua}
\title[Linear Upper Bound on a Segment XOR Cardinality]{Linear Upper Bound on a Segment XOR Cardinality}
\date{\today}

\begin{document}

\begin{frame}
    \titlepage
\end{frame}

\begin{frame}{Introduction}
    \textbf{Bitwise XOR} (exclusive or) is a fundamental operation in theoretical computer science.
    \begin{itemize}
        \item We study properties of XOR applied to sets.
        \item Defined as $X \oplus Y = \{ x \oplus y \mid x \in X, y \in Y \}$.
        \item Focus on XORs of segments of consecutive integers.
    \end{itemize}
\end{frame}

\begin{frame}{Problem Statement}
    \textbf{Segment Notation:} $[x, x+k) = \{x, x+1, \dots, x+k-1\}$.
    \begin{block}{Conjecture}
        $|[x, x+k) \oplus [y, y+k)| \leq 4(k-1)$ for any positive integer $k$ and nonnegative integers $x, y$.
    \end{block}
    \textbf{Experimental Observations:} Holds for all $k \leq 2^9 + 2$.
\end{frame}

\begin{frame}{Key Remark}
    \begin{block}{Tight Bound}
        The bound is tight for infinitely many values of $k$.
        \begin{itemize}
            \item Example: $k = 2^m + 2$.
            \item Values: $x = 2^m - 1$, $y = 3 \cdot 2^m$.
        \end{itemize}
    \end{block}
\end{frame}

\begin{frame}{Main Result}
    \begin{block}{Stronger Bound}
        Our result improves on the naive quadratic upper bound:
        \[|X \oplus Y| \leq |X| \cdot |Y| = k^2.\]
    \end{block}
    \begin{block}{Theorem}
        $|[x, x+k) \oplus [y, y+k)| \leq 5(k-2)$ for $k \geq 5$.
    \end{block}
\end{frame}

\begin{frame}{Proof Strategy}
    \begin{itemize}
        \item Base cases are established computationally.
        \item Recursive bounds:
        \begin{align*}
            f(2k) &\leq 2f(k+1),\\
            f(2k+1) &\leq 2f(k+1).
        \end{align*}
        \item Inductive proof follows from these inequalities.
    \end{itemize}
\end{frame}

\begin{frame}{Key Lemmas}
    \begin{lemma}
        For any fixed $k$, the optimization problem \[ g(k; x, y) = |[x, x+k) \oplus [y, y+k)| \to \max_{x, y} \] has an optimal solution $(x_0, y_0)$ with $x_0, y_0 \leq 4k$.
    \end{lemma}
    \begin{lemma}
        If $f(k) = \max_{x, y} g(k; x, y)$, then:
        \[ f(2k) \leq 2f(k+1), \quad f(2k+1) \leq 2f(k+1). \]
    \end{lemma}
\end{frame}

\begin{frame}{Conclusion}
    \begin{itemize}
        \item Established a linear upper bound on segment XOR cardinality.
        \item Stronger than naive $O(k^2)$ bound.
        \item Future work: Refinement of constant factors.
    \end{itemize}
\end{frame}

\begin{frame}{References}
    \begin{thebibliography}{9}
        \bibitem{stanley} R. P. Stanley. \emph{Enumerative Combinatorics}. Cambridge University Press, 1997.
        \bibitem{bach} E. Bach, J. Shallit. \emph{Algorithmic Number Theory}. MIT Press, 1996.
        \bibitem{sedgewick} R. Sedgewick, K. Wayne. \emph{Algorithms}. Addison-Wesley, 2011.
    \end{thebibliography}
\end{frame}


\end{document}
