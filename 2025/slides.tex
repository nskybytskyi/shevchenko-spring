\documentclass[handout,notheorems]{beamer}

\usepackage[T2A,T1]{fontenc}
\usepackage[utf8]{inputenc}
\usepackage[english]{babel}

\usepackage{tikz}
\usepackage{amsthm}
\newtheorem*{lemma}{Lemma}
\theoremstyle{definition}
\newtheorem*{definition}{Definition}

\hypersetup{unicode=true}

\usetheme{CambridgeUS}

\setbeamercolor{frametitle}{parent=palette primary}

\setbeamercolor{block title}{fg=black,bg=white!60!gray}
\setbeamercolor{block body}{fg=black,bg=white!90!gray}

\setbeamercolor*{author in head/foot}{parent=palette tertiary}
\setbeamercolor*{title in head/foot}{parent=palette secondary}
\setbeamercolor*{date in head/foot}{parent=palette primary}

\setbeamercolor*{section in head/foot}{parent=palette tertiary}
\setbeamercolor*{subsection in head/foot}{parent=palette primary}

\setbeamercolor{title}{bg=red!65!black,fg=white}
\setbeamersize{text margin left=1em,text margin right=1em}

% \AtBeginSection[]
% {
%     \begin{frame}{Table of Contents}
%         \tableofcontents[currentsection,hideothersubsections]
%     \end{frame}
% }

\institute[KNU]{Taras Shevchenko National University of Kyiv}
\author[Skybytska N.]{Nika Skybytska, n.skybytskyi@knu.ua}
\title[Linear Upper Bound on Segment XOR]{Linear Upper Bound on a Segment XOR Cardinality}
\date{\today}

\begin{document}

\begin{frame}
    \titlepage
\end{frame}

\section{Introduction}

\subsection{Definitions}

\begin{frame}
    \textbf{Bitwise XOR} (exclusive or) is a fundamental operation in theoretical computer science. Example: $10 \oplus 3$ (in binary):
    \begin{equation*}
        \begin{array}{rl}
            10 &= 1010_2 \\
             3 &= 0011_2 \\
            \hline
            10 \oplus 3 &= 1001_2 = 9
        \end{array}
    \end{equation*}
    \pause
    \begin{definition}
        We study properties of XOR applied to sets, defined as $X \oplus Y = \{ x \oplus y \mid x \in X, y \in Y \}$.
    \end{definition}
\end{frame}

\subsection{Examples}

\begin{frame}
    Let $X = \{1, 2\}$ and $Y = \{3, 4\}$. $X \oplus Y = \{1, 2, 5, 6\}$:
    \begin{table}
        \begin{tabular}{c|cc}
            $\oplus$ & 3 & 4 \\ \hline
            1 & 2 & 5 \\
            2 & 1 & 6 \\
        \end{tabular}
    \end{table}
    \pause
    Let $X = \{1, 2, 3\}$ and $Y = \{2, 3, 4\}$. $X \oplus Y = \{0, 1, 2, 3, 5, 6, 7\}$:
    \begin{table}
        \begin{tabular}{c|ccc}
            $\oplus$ & 2 & 3 & 4 \\ \hline
            1 & 3 & 2 & 5 \\
            2 & \textbf{0} & \textit{1} & 6 \\
            3 & \textit{1} & \textbf{0} & 7 
        \end{tabular}
    \end{table}
\end{frame}

\section{Methods}

\subsection{Problem Statement}

\begin{frame}
    \begin{definition}
        Focus on XORs of segments of consecutive integers: $[l, r) = \{l, l+1, \dots, r-1\}$.
    \end{definition}
    \pause
    \begin{block}{Problem}
        Given a positive integer $k$, solve $|[x, x+k) \oplus [y, y+k)| \to \max_{x,y}$ over nonnegative integers $x, y$.
    \end{block}
    \pause
    \begin{block}{Remark 1}
        Two subproblems: establish an \textbf{upper bound} and provide an efficient \textbf{construction}. Both experimental and analytical methods are fine.
    \end{block}
\end{frame}

\section{Results}

\subsection{Main Result}

\begin{frame}
    \begin{block}{Theorem}
        $|[x, x+k) \oplus [y, y+k)| \le 5(k-2)$ for all $k \ge 5$.
    \end{block}
    \pause
    \begin{block}{Remark 2}
        Our result improves on the naive quadratic upper bound:
        \begin{equation}
            |X \oplus Y| \le |X| \cdot |Y| = k^2.
        \end{equation}
    \end{block}
\end{frame}

\subsection{Conjecture}

\begin{frame}
    \begin{block}{Proposition}
        $|[x, x+k) \oplus [y, y+k)| \le 4(k-1)$ for all $k > 1$.
    \end{block}
    \pause
    \begin{block}{Remark 3}
        This bound is tight for infinitely many values of $k = 2^m + 2$. The construction is $x = 2^m - 1$, $y = 3 \cdot 2^m$.
    \end{block}
\end{frame}

\section{Discussion}

\subsection{Proof Sketch}

\begin{frame}
    \begin{lemma}
        For any fixed $k$, the optimization problem 
        \begin{equation}
            g(k; x, y) = |[x, x+k) \oplus [y, y+k)| \to \max_{x, y}
        \end{equation}
        has an optimal solution $(x_0, y_0)$ with $x_0, y_0 \le 4k$.
    \end{lemma}
    \pause
    \begin{block}{Corollary}
        Hence, base cases can be established computationally.
    \end{block}
\end{frame}

\subsection{Proof Strategy}

\begin{frame}
    \begin{lemma}
        If $f(k) = \max_{x, y} g(k; x, y)$, then:
        \begin{equation}
            f(2k) \le 2f(k+1), \quad f(2k+1) \le 2f(k+1).
        \end{equation}
    \end{lemma}
    \pause
    \begin{block}{Proof}
        Inductive proof follows from these inequalities:
        \begin{equation}
            \begin{aligned}
                f(2k) &\le 2f(k+1) \le 5 (2k - 2) \\
                f(2k+1) &\le 2f(k+1) \le 5 (2k - 1).
            \end{aligned}
        \end{equation}
    \end{block}
\end{frame}

\subsection{Conclusions}

\begin{frame}
    \begin{itemize}
        \item We established a linear upper bound on segment XOR cardinality.
        \item It is stronger than naive $O(k^2)$ bound.
        \item Future work: refinement of constant factors.
    \end{itemize}
\end{frame}

\begin{frame}{References}
    \begin{thebibliography}{9}
        \bibitem{stanley} R. P. Stanley. \emph{Enumerative Combinatorics}. Cambridge University Press, 1997.
        \bibitem{bach} E. Bach, J. Shallit. \emph{Algorithmic Number Theory}. MIT Press, 1996.
        \bibitem{sedgewick} R. Sedgewick, K. Wayne. \emph{Algorithms}. Addison-Wesley, 2011.
    \end{thebibliography}
\end{frame}

\end{document}
