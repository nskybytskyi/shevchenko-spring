\documentclass[ascii]{abook}

\begin{document}

\begin{abstract}

\selectlanguage{english}

\title{Linear Upper Bound on a Segment XOR Cardinality}

\author[N.~Skybytskyi]{Nikita M. Skybytskyi}
{Taras Shevchenko National University of Kyiv, Kyiv, Ukraine.}
{n.skybytskyi@knu.ua}

\maketitle

In theoretical computer science, bitwise XOR (exclusive or) is a fundamental binary operation on nonnegative integers. We study the properties of bitwise XOR of sets, extending concepts from combinatorial set theory \cite{skybytskyi:stanley_enumerative}. Namely, for two sets $X$ and $Y$ of nonnegative integers, we denote $X \oplus Y = \{x \oplus y \mid x \in X, y \in Y\}$.

We focus on XORs of segments of consecutive integers, leveraging insights from algorithmic number theory \cite{skybytskyi:bach_algorithmic}. We use a shorthand notation: $[x, x+k) = \{x, x+1, \dots, x+k-1\}$ for a nonnegative integer $x$ and a positive integer $k$. The following result emerged experimentally:

\begin{conjecture*}
    $|[x, x+k) \oplus [y, y+k)| \le 4(k-1)$ for any positive integer $k$ and nonnegative integers $x, y$.
\end{conjecture*}

\begin{remark*}
    This bound is tight for infinitely many values of $k$. One series of particular interest is $k = 2^m + 2$ with $x = 2^m - 1$ and $y = 3 \cdot 2^m$.
\end{remark*}

This linear upper bound is much stronger than a naive quadratic upper bound of $|X \oplus Y| \le |X| \cdot |Y| = k^2$. Even though we verified it computationally for all $k \le 2^9 + 2$, we ultimately failed to prove it rigorously. However, we managed to produce a slightly weaker result:

\begin{theorem*}
    $|[x, x+k) \oplus [y, y+k)| \le 5(k-2)$ for any positive integer $k \ge 5$ and any nonnegative integers $x, y$.
\end{theorem*}
% \begin{remark*}
%     The practical significance of this weaker result is that one cannot consider segments of consecutive integers when looking for xor-combinations with a large span since the resulting size is only $O(k)$ when the maximum for arbitrary sets is $\Theta(k^2)$.
% \end{remark*}

The following lemmas are central to the proof:

\begin{lemma}
    For any fixed $k$, the optimization problem $g(k; x, y) = |[x, x+k) \oplus [y, y+k)| \to \max_{x, y}$ has an optimal solution $(x_0, y_0)$ with $x_0, y_0 \le 4k$.
\end{lemma}

\begin{lemma}
    If we denote $f(k) = \max_{x, y} g(k; x, y)$ then two inequalities hold: $f(2k) \le 2f(k+1)$ and $f(2k+1) \le 2f(k+1)$, as inspired by \cite{skybytskyi:sedgewick_algorithms}.
\end{lemma}

The proof of our main result proceeds by induction with the first lemma establishing base cases and the second lemma helping with inductive steps.
% \begin{remark*}
%     According to the master theorem, the second lemma alone proves that $f(k) = O(k)$. However, it offers no specific bounds on the constant hidden in big-O notation.
% \end{remark*}


\begin{thebibliography}{9}
\bibitem{skybytskyi:stanley_enumerative} R. P. Stanley. \emph{Enumerative Combinatorics}. Cambridge University Press, 1997.
\bibitem{skybytskyi:bach_algorithmic} E. Bach and J. Shallit. \emph{Algorithmic Number Theory: Efficient Algorithms}. MIT Press, 1996.
\bibitem{skybytskyi:sedgewick_algorithms} R. Sedgewick and K. Wayne. \emph{Algorithms}. Addison-Wesley, 2011.
\end{thebibliography}

\end{abstract}

\end{document}
